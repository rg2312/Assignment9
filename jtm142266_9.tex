%%%%%%%%%%%%%%%%%%%%%%%%%%%%%%%%%%%%  ASSIGNMENT 9 %%%%%%%%%%%%%%%%%%%%%%%%%%%%%%%%%%%%%%%%%%%%%%%%%%%%%%%%%%
%%%%%%%%%%%%%%%%%%%%%%%%%%%%%%%%%%%%  NAME: ROHIT GARG          %%%%%%%%%%%%%%%%%%%%%%%%%%%%%%%%%%%%%%%%%%%%%
%%%%%%%%%%%%%%%%%%%%%%%%%%%%%%%%%%%%  ENTRY NUMBER:2014JTM2266   %%%%%%%%%%%%%%%%%%%%%%%%%%%%%%%%%%%%%%%%%%%%

\documentclass[a4paper,10pt]{article}
\usepackage{fontenc}
\usepackage{graphicx} % to include images
\usepackage[top=1in,bottom=1in,left=1in,right=1in]{geometry} % to set page margins 
\usepackage{url} % to include url's
\usepackage{color} % to include colors
\usepackage{fancyhdr} % for header and footer


\date{} % no date is diplayed out because argument is left blank


\begin{document} % marks the beginning of the document
\begin{titlepage} % marks the beginning of the title page
\begin{center} % to center text/images

\huge \textbf{ \textcolor{blue}{\rule{\textwidth}{8pt}}\\ EEP 773 \\[1cm] Telecom Software Lab\\ \textcolor{blue}{\rule{\textwidth}{8pt}} 
\\[3cm] Assignment 9 
}\\[1cm] 24 September 2014 \\[3cm] \textbf{ROHIT GARG} \\[1cm] 2014JTM2266\\[3cm] % [2cm] etc specifies the vertical spacing
% increased sized font as compared to normal fonts and in bold font series
\end{center}
\begin{figure}[h]
 \centering
\includegraphics[bb=0 0 2000 2000]{1.png}
 % 1.png: 2000x2000 pixel, 72dpi, 70.55x70.55 cm, bb=0 0 2000 2000
 \huge \bfseries \underline{Indian Institute of Technology, Delhi}
\end{figure}


\end{titlepage} % title page ends

\newpage % new page starts
\tableofcontents % for writing table of contents
\thispagestyle{empty} % here to remove page number from the table of contents page
\newpage % new page starts
\listoffigures % to include the list of figures
\thispagestyle{empty} % here to remove page number from the table of contents page
\newpage % new page starts

\setcounter{page}{1} % sets page number to the current page as 1 and starts the counter onwards for succeeding pages
\pagestyle{fancy} % sets page style as fancy to include header and footer
\fancyhf{}
\cfoot{ \emph{ \scriptsize{EEP Telecom Software Lab: Assignment 9 | Rohit Garg | 24 September 2014}}} % footer text
\rfoot{\thepage}
\renewcommand{\headrulewidth}{0pt} % to diminish the header rule line

%%%%%%%%%%%%%%%%%%%%%%%%%%%%%%%%%%%%%%%%%%%%%%%%%%%%%%%%%  PROBLEM STATEMENT  %%%%%%%%%%%%%%%%%%%%%%%%%%%%%%%%%%%%%%%%%%%%%%%%%%%%%%%%%%%%%%%%%%%%%%%%%%%%


% \section{} and subsection{} are used to list them in table of contents and systematically number them accordingly
\section{Problem Statement}

This assignment is to use Python and git.

We have to make a code that searches emoticons through a given file and the percentage of the number of times the emotion occured.\linebreak
We have to print the mood of the $ user\_1$ available there in the file in the following format
$$ user\_1\ :\ @user\_2 \      statement $$

file from \url https://github.com/shubhras01/Assignment9.git can be taken

The ouput should be printed in a file. Also you need to add your files back on the github online reppository created by you by the name $Assignment\_9$.
%%%%%%%%%%%%%%%%%%%%%%%%%%%%%%%%%%%%%%%%%%%%%%%%%%%%%%%%%%%% ASSUMPTIONS  %%%%%%%%%%%%%%%%%%%%%%%%%%%%%%%%%%%%%%%%%%%%%%%%%%%%%%%%%%%%%%%%%%%%%%%%%%%%%%
\newpage   % new page starts
\section{Assumptions}
I have taken following assumptions while preparing solution to the given problem statement :
\begin{enumerate}
 \item The code is limited for the givenfile.
 \item The order can not be chnanged for any user in the file.
 
\end{enumerate}

%%%%%%%%%%%%%%%%%%%%%%%%%%%%%%%%%%%%%%%%%%%%%%%%%%%%%%%%%%%%%%% LOGIC EXPLANATION %%%%%%%%%%%%%%%%%%%%%%%%%%%%%%%%%%%%%%%%%%%%%%%%%%%%%%%%%%%%%%%%%%%%%%
\newpage   % new page starts
\section{Logic Explanation} 

I have divided the complete problem statement into various modules and tried to understand, code and analyze these; collaborating
to form the answer to the given problem statement.
The complete logic and module description are as below :

\subsection{The Python Code}
\subsubsection{TASK 1 - Mood of the user}

The code made includes first opening the file using $open()$ function a read mode.
I have created a list for each of the 5 users viz A, B, C, E, G so as to count corresponding emotions associated with them.
Using a function $line.split()$ the complete line fetched through the file is broken into words which are then compared with the available emotions
and on matching the lists are getting updated.

A function/method named $maximum()$ is called in which all the list elements are compared with one another and the index wih maximum number associated 
with it is pulled out to return the mood.

The function is called for every user viz A, B, C, E, G.

\subsection{TASK 2 - Printing percentage of different moods}

The complete file opened again to scan each line here too and directly all the emotion symbols are compared with the words and the values
are stored in a counter created for seperate emotion namely $emoHappy$, $emoSad$ etc.

Then using simple mathematics and $print$ command the result is displayed out.

\section{Using Git}

Here, firstly all the necessary file are forked from the link provided above in the problem statement into my github account.
From there these files are cloned into the system local repository.

%%%%%%%%%%%%%%%%%%%%%%%%%%%%%%%%%%%%%%%%%%%%%%%%%%%%%%%%%%  SCREENSHOTS  %%%%%%%%%%%%%%%%%%%%%%%%%%%%%%%%%%%%%%%%%%%%%%%%%%%%%%%%%%%%%%%%%%%%%%%%%%%%%%%%%%
\newpage % new page starts
\section{Screenshots}
\subsection{Screenshot 1 - Showing Final Output on Terminal}
\begin{figure}[h]
 \centering
 \includegraphics[bb=0 0 816 337,scale=0.5]{./2.png}
 % 2.png: 816x337 pixel, 72dpi, 28.79x11.89 cm, bb=0 0 816 337
\end{figure}

\subsection{Screenshot 2 - Showing Final Output in a File}



%%%%%%%%%%%%%%%%%%%%%%%%%%%%%%%%%%%%%%%%%%%%%%%%%%%%%%%%%%  BIBLIOGRAPHY  %%%%%%%%%%%%%%%%%%%%%%%%%%%%%%%%%%%%%%%%%%%%%%%%%%%%%%%%%%%%%%%%%%%%%%%%%%%%%%%
\newpage % new page starts
\begin{thebibliography}{widestlabel} % bibliography beginning
\bibitem{}\url{http://www.tutorialspoint.com/python/}          %\bibitem is for bibliography item and \url to write url's
\bibitem{}\url{https://www.youtube.com/watch?v=4dVtFLkpRjc}
\bibitem{}\url{http://tex.stackexchange.com/questions}
\bibitem{}\url{https://www.sharelatex.com/learn}

\end{thebibliography} % ends the bibliography
 
\end{document} % marks the end of the document

%%%%%%%%%%%%%%%%%%%%%%%%%%%%%%%%%%%%%%%%%%%%%%%%%%%%%%%%%%%%%%%%%%%%%%%%%%%%%%%%%%%%%%%%%%%%%%%%%%%%%%%%%%%%%%%%%%%%%%%%%%%%%%%%%%%%%%%%%%%%%%%%%%%%%%%